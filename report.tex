% !Mode:: "TeX:UTF-8"
%%%%%%%%%%%%%%%%%%%%%%%%%%%%%%%%%%%%%%%%%%%%%%%%%%%%%%%%%%%%%%%%%%%%%%%%%%%%%%%%
%          ,
%      /\^/`\
%     | \/   |                CONGRATULATIONS!
%     | |    |             SPRING IS IN THE AIR!
%     \ \    /                                                _ _
%      '\\//'                                               _{ ' }_
%        ||                     hithesis v3                { `.!.` }
%        ||                                                ',_/Y\_,'
%        ||  ,                   dustincys                   {_,_}
%    |\  ||  |\          Email: yanshuoc@gmail.com             |
%    | | ||  | |            https://yanshuo.name             (\|  /)
%    | | || / /                                               \| //
%    \ \||/ /       https://github.com/dustincys/hithesis      |//
%      `\\//`   \\   \./    \\ /     //    \\./   \\   //   \\ |/ /
%     ^^^^^^^^^^^^^^^^^^^^^^^^^^^^^^^^^^^^^^^^^^^^^^^^^^^^^^^^^^^^^^
%
% ----------------------------- Warning -----------------------------
% 注意: 这不是 GitHub 原版
% Modification:
%   1. 添加了页眉
%   2. 修改了目录格式以匹配中期模板
%   3. 修改了封面以匹配中期模板
%   4. 添加了 联合导师 和 副导师 选项
%   5. 取消了 latexmk 的连续编译选项
%   6. 将编译输出文件夹修改为了 ./build, build 中需要建立和文字目录对应的子文件夹
%   6. 添加了多种列表环境
%   7. 将公式编号修改为了 (2-1) 的形式
%   8. 在参考文献卡你添加了 \FloatBarrier
%   9. 添加了等宽表格例子
% TexStudio 使用
%   1. 将编译器改成 latexmk
%   2. 将输出文件夹改为 build
% 2021-05-24 10:33:44 by ruic
% ----------------------------- Warning -----------------------------
%
%%%%%%%%%%%%%%%%%%%%%%%%%%%%%%%%%%%%%%%%%%%%%%%%%%%%%%%%%%%%%%%%%%%%%%%%%%%%%%%%
\documentclass[fontset=windows,toc=true,type=master,stage=midterm,campus=shenzhen]{hithesisart}
% 此处选项中不要有空格
%%%%%%%%%%%%%%%%%%%%%%%%%%%%%%%%%%%%%%%%%%%%%%%%%%%%%%%%%%%%%%%%%%%%%%%%%%%%%%%%
% 必填选项
% type=doctor|master|bachelor
% stage=opening|midterm
%%%%%%%%%%%%%%%%%%%%%%%%%%%%%%%%%%%%%%%%%%%%%%%%%%%%%%%%%%%%%%%%%%%%%%%%%%%%%%%%
% 选填选项(选填选项的缺省值已经尽可能满足了大多数需求,除非明确知道自己有什么
% 需求)
% campus=shenzhen|weihai|harbin
%   含义:校区选项,默认harbin
% fontset=windows|mac|ubuntu|fandol
%   含义:前三个对应各自系统,fandol是开源字体。
%%%%%%%%%%%%%%%%%%%%%%%%%%%%%%%%%%%%%%%%%%%%%%%%%%%%%%%%%%%%%%%%%%%%%%%%%%%%%%%%


\graphicspath{{figures/}{logos/}}


\begin{document}

% !Mode:: "TeX:UTF-8"

\hitsetup{
  %******************************
  % 注意:
  %   1. 配置里面不要出现空行
  %   2. 不需要的配置信息可以删除
  %******************************
  ctitlecover={局部多孔质气体静压轴承关键技术的研究},%放在封面中使用,自由断行
  % ctitleone={局部多孔质气体静压},
  % ctitletwo={轴承关键技术的研究},
  caffil={机电工程学院},
  csubject={机械制造及其自动化},
  cauthor={于冬梅},
  cstudentid={9527},
  cclassid={9527},
  csupervisor={某某某教授},
  % 日期自动使用当前时间,若需指定按如下方式修改:
  %cdate={盘古开天地}
  % cenrolldate={公元2020年}
}

\makecover


% 深圳博士开题报告 -----------------------------------------------
\pagestyle{hit@headingsmunual} % 页眉
% section 后有*表示不需要画横线
\section{论文工作是否按开题报告预定的内容及进度安排进行}*

\lipsum[1]

\section{目前已完成的研究工作及结果}

\subsection{二级标题}
\subsubsection{三级标题}
\paragraph{四级标题}
\lipsum[2]
\paragraph{四级标题}

\begin{equation}
    H_{\epsilon}(z)=\frac{1}{2}\left[1+\frac{2}{\pi} \arctan \left(\frac{z}{\epsilon}\right)\right] .
\end{equation}

\lipsum[3]

\begin{table}[htbp]
    \caption{页面宽度表格。}
    \newcolumntype{Y}{>{\centering\arraybackslash}X}
    % 仍然可以使用 p{3em} 指定某列宽度, 并保持整个表格宽度,从而适应不等宽列
    \begin{tabularx}{\textwidth}{*{5}{Y}}
        \toprule
             & Dice   & JS     & BF     & PN     \\
        \midrule
        Mean & 0.9999 & 0.9999 & 0.9999 & 0.9999 \\
        Max  & 0.9999 & 0.9999 & 0.9999 & 0.9999 \\
        Min  & 0.9999 & 0.9999 & 0.9999 & 0.9999 \\
        Std. & 0.9999 & 0.9999 & 0.9999 & 0.9999 \\
        \bottomrule
    \end{tabularx}
    \label{tab: quantitatively evaluation results}
\end{table}

\subsection{二级标题}
\subsubsection{三级标题}
\paragraph{四级标题}
\lipsum[2]
\paragraph{四级标题}
\lipsum[3]

\section{后期拟完成的研究工作及进度安排}

\subsection{subsection title}
\subsubsection{subsection title}
\lipsum[2-3]
\subsection{subsection title}
\subsubsection{subsection title}
\lipsum[4-5]

\section{存在的困难与问题}

\subsection{subsection title}
\lipsum[2-3]
\subsection{subsection title}
\lipsum[4-5]

\section{如期完成全部论文工作的可能性}*

\subsection{subsection title}
\lipsum[2-3]
\subsection{subsection title}
\lipsum[4-5]

\clearpage
\section*{指导教师意见:}

\begin{figure}[h]
    \centering
    \includegraphics[width=0.5\textwidth]{快逃.jpg}
\end{figure}

\vspace{24bp}
\hfill 指导教师签名:\hspace{8\ccwd}

\vspace{24bp}
\hfill 年\hspace{2\ccwd}月\hspace{2\ccwd}日

\vspace{24bp}
\section*{检查小组意见:}

\vspace{80bp}
\hfill 组长(签字):\hspace{8\ccwd}

\vspace{24bp}
\hfill 年\hspace{2\ccwd}月\hspace{2\ccwd}日


% 威海校区本科生开题报告结构 ----------------------------------------
% % !Mode:: "TeX:UTF-8"
\section{课题背景及研究的目的和意义}
\subsection{课题背景}
(正文  宋体小4号字,多倍行距值1.25,段前0行,段后0行。字数3000字以上。具体的撰写要符合哈尔滨工业大学本科生毕业论文撰写规范的书写规定。)\cite{hithesis2017}\inlinecite{cnproceed}
\subsection{研究的目的和意义}
\section{国内外在该方向的研究现状及分析}
\subsection{国外现状及分析}
\subsection{国内现状及分析}
\section{研究内容及拟解决的关键问题}
\subsection{研究内容}
\subsection{拟解决的关键问题}
\section{拟采取的研究方法和技术路线、进度安排、预期达到的目标}
\subsection{拟采取的研究方法和技术路线}
\subsection{进度安排}
\subsection{预期达到的目标}
\section{课题已具备和所需的条件}
\section{研究过程中可能遇到的困难和问题,解决的措施}
\section{参考文献}
\bibliographystyle{hithesis}
\bibliography{reference}

% Local Variables:
% TeX-master: "../report"
% TeX-engine: xetex
% End:
% \makebackcover
% -------------------------------------------------------------

% 威海校区本科生中期报告结构 ----------------------------------------
% \section{论文工作是否按预期进行、目前已完成的研究工作及结果}
\subsection{论文工作是否按预期进行}
(正文  宋体小4号字,多倍行距值1.25,段前0行,段后0行。字数3000字以上。具体的撰写要符合哈尔滨工业大学本科生毕业论文撰写规范的书写规定。)
\subsection{目前已完成的研究工作及结果}
\section{后期拟完成的研究工作及进度安排}
\subsection{后期拟完成的研究工作}
\subsection{后期进度安排}
\section{存在的问题与困难}
\section{论文按时完成的可能性}
\section{参考文献}
\bibliographystyle{hithesis}
\bibliography{reference}

% Local Variables:
% TeX-master: "../mainart"
% TeX-engine: xetex
% End:
% \makebackcover
% -------------------------------------------------------------

% 哈尔滨校区本科生开题报告结构 --------------------------------------
% \section{课题来源及研究的目	的和意义}
(正文  宋体小4号字,行距1.25倍,段前0行,段后0行)
\section{国内外在该方向的研究现状及分析}
\section{主要研究内容}
\section{研究方案}
\section{进度安排,预期达到的目标}
\section{课题已具备和所需的条件、经费}
\section{研究过程中可能遇到的困难和问题,解决的措施}
\section{主要参考文献}
\bibliographystyle{hithesis}
\bibliography{reference}

% Local Variables:
% TeX-master: "../mainart"
% TeX-engine: xetex
% End:
% -------------------------------------------------------------

\FloatBarrier
\section*{参考文献}
\bibliographystyle{hithesis}
\bibliography{reference}
% inline style \inlinecite, upper style \cite

\end{document}

% Local Variables:
% TeX-engine: xetex
% End:
