% * =========================================================================
% * 课题主要研究内容及进度
% * =========================================================================
\section{课题主要研究内容及进度}

% * -------------------------------------------------------------------------
% * 课题主要研究内容
% * -------------------------------------------------------------------------
\subsection{课题主要研究内容}

\lipsum[1]

% * -------------------------------------------------------------------------
% * 进度介绍
% * -------------------------------------------------------------------------
\subsection{进度介绍}

\lipsum[2]

% * =========================================================================
% * 目前已完成的主要研究工作及结果
% * =========================================================================
\section{目前已完成的主要研究工作及结果}

\subsection{二级标题}

\subsubsection{三级标题}

\paragraph{四级标题}

\lipsum[2]

\paragraph{公式与图表}

式\eqref{eq: example} 是一个公式示例。表 \ref{tab: table example} 是一个表格实例。图 \ref{fig: figure example} 是一个插图示例。

\begin{equation} \label{eq: example}
    H_{\epsilon}(z)=\frac{1}{2}\left[1+\frac{2}{\pi} \arctan \left(\frac{z}{\epsilon}\right)\right] .
\end{equation}

\begin{table}[htbp]
    \caption{页面宽度表格}
    \newcolumntype{Y}{>{\centering\arraybackslash}X}
    % 仍然可以使用 p{3em} 指定某列宽度, 并保持整个表格宽度,从而适应不等宽列
    \begin{tabularx}{\textwidth}{*{5}{Y}}
        \toprule
             & Dice   & JS     & BF     & PN     \\
        \midrule
        Mean & 0.9999 & 0.9999 & 0.9999 & 0.9999 \\
        Max  & 0.9999 & 0.9999 & 0.9999 & 0.9999 \\
        Min  & 0.9999 & 0.9999 & 0.9999 & 0.9999 \\
        Std. & 0.9999 & 0.9999 & 0.9999 & 0.9999 \\
        \bottomrule
    \end{tabularx}
    \label{tab: table example}
\end{table}

\begin{figure}[htbp]
    \centering
    \includegraphics[width=0.5\textwidth]{快逃.jpg}
    \caption{示例图片}
    \label{fig: figure example}
\end{figure}

\paragraph{参考文献}
当文献做主语时,需要将文献放在行内,使用 \verb|\inlinecite{cnproceed}|,如:文献 \inlinecite{cnproceed} 中描述了 XXXX。
其他情况下使用 \verb|\cite{cnproceed}| 如:XXXX\cite{cnproceed}。

\subsection{二级标题}
\subsubsection{三级标题}
\paragraph{四级标题}
\lipsum[2]
\paragraph{四级标题}
\lipsum[3]

% * =========================================================================
% * 后期拟完成的研究工作及进度安排
% * =========================================================================
\section{后期拟完成的研究工作及进度安排}

本课题的所定目标已经完成了大部分,后期需要完成的工作包括 XXXXXX,具体工作为:
\begin{semiQuotList}
    \item 工作 1。
    \item 工作 2。
    \item 工作 3。
\end{semiQuotList}

具体工作进度安排为:
\begin{closeItemize}
    \item 202X 年 X 月 -- 202X 年 X 月:XXXXX。
    \item 202X 年 X 月 -- 202X 年 X 月:将牙齿方向判断合并到牙齿检测模型中,并尝试使用 Anchor Free 检测方法。
    \item 202X 年 X 月 -- 202X 年 X 月:尝试将活动轮廓处理流程合并到深度学习模型内部。
    \item 202X 年 X 月 -- 202X 年 X 月:整理结果,撰写毕业论文,准备毕业答辩。
\end{closeItemize}

% * =========================================================================
% * 存在的困难与问题
% * =========================================================================
\section{存在的困难与问题}
本课题涉及 XXXXXX。此外,XXXXXXX,给课题的实现带来了一定的困难。总体上,预期可能遇到的困难有:
\begin{semiQuotList}
    \item 问题 1。
    \item 问题 2。
    \item 问题 3。
    \item 问题 4。
    \item 问题 5。
\end{semiQuotList}

% * -------------------------------------------------------------------------
% * 存在问题与困难
% * -------------------------------------------------------------------------
\subsection{存在问题与困难}
\lipsum[4-5]

% * -------------------------------------------------------------------------
% * 解决方案
% * -------------------------------------------------------------------------
\subsection{解决方案}
为了 XXXXXXX。具体来说,有以下解决方案:
\begin{semiQuotList}
    \item 解决方法 1。
    \item 解决方法 2。
    \item 解决方法 3。
    \item 解决方法 4。
    \item 解决方法 5。
\end{semiQuotList}

% * =========================================================================
% * 如期完成全部论文工作的可能性
% * =========================================================================
\section{如期完成全部论文工作的可能性}

本课题目前基本 XXXXXX,已经完成了  XXXXXX 构建。
目前正在进行 XXXXXXX,将要进行 XXXXX。
综上所述,论文能够按期完成,并取得一定的研究成果。