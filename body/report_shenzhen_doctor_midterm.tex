% section 后有*表示不需要画横线
\section{论文工作是否按开题报告预定的内容及进度安排进行}*

\lipsum[1]

\section{目前已完成的研究工作及结果}

\subsection{二级标题}
\subsubsection{三级标题}
\paragraph{四级标题}
\lipsum[2]
\paragraph{四级标题}

\begin{equation}
    H_{\epsilon}(z)=\frac{1}{2}\left[1+\frac{2}{\pi} \arctan \left(\frac{z}{\epsilon}\right)\right] .
\end{equation}

\lipsum[3]

\begin{table}[htbp]
    \caption{页面宽度表格。}
    \newcolumntype{Y}{>{\centering\arraybackslash}X}
    % 仍然可以使用 p{3em} 指定某列宽度, 并保持整个表格宽度,从而适应不等宽列
    \begin{tabularx}{\textwidth}{*{5}{Y}}
        \toprule
             & Dice   & JS     & BF     & PN     \\
        \midrule
        Mean & 0.9999 & 0.9999 & 0.9999 & 0.9999 \\
        Max  & 0.9999 & 0.9999 & 0.9999 & 0.9999 \\
        Min  & 0.9999 & 0.9999 & 0.9999 & 0.9999 \\
        Std. & 0.9999 & 0.9999 & 0.9999 & 0.9999 \\
        \bottomrule
    \end{tabularx}
    \label{tab: quantitatively evaluation results}
\end{table}

\subsection{二级标题}
\subsubsection{三级标题}
\paragraph{四级标题}
\lipsum[2]
\paragraph{四级标题}
\lipsum[3]

\section{后期拟完成的研究工作及进度安排}

\subsection{subsection title}
\subsubsection{subsection title}
\lipsum[2-3]
\subsection{subsection title}
\subsubsection{subsection title}
\lipsum[4-5]

\section{存在的困难与问题}

\subsection{subsection title}
\lipsum[2-3]
\subsection{subsection title}
\lipsum[4-5]

\section{如期完成全部论文工作的可能性}*

\subsection{subsection title}
\lipsum[2-3]
\subsection{subsection title}
\lipsum[4-5]

\clearpage
\section*{指导教师意见:}

\begin{figure}[h]
    \centering
    \includegraphics[width=0.5\textwidth]{快逃.jpg}
\end{figure}

\vspace{24bp}
\hfill 指导教师签名:\hspace{8\ccwd}

\vspace{24bp}
\hfill 年\hspace{2\ccwd}月\hspace{2\ccwd}日

\vspace{24bp}
\section*{检查小组意见:}

\vspace{80bp}
\hfill 组长(签字):\hspace{8\ccwd}

\vspace{24bp}
\hfill 年\hspace{2\ccwd}月\hspace{2\ccwd}日
